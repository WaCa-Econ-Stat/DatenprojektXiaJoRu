% Options for packages loaded elsewhere
\PassOptionsToPackage{unicode}{hyperref}
\PassOptionsToPackage{hyphens}{url}
%
\documentclass[
]{article}
\usepackage{amsmath,amssymb}
\usepackage{iftex}
\ifPDFTeX
  \usepackage[T1]{fontenc}
  \usepackage[utf8]{inputenc}
  \usepackage{textcomp} % provide euro and other symbols
\else % if luatex or xetex
  \usepackage{unicode-math} % this also loads fontspec
  \defaultfontfeatures{Scale=MatchLowercase}
  \defaultfontfeatures[\rmfamily]{Ligatures=TeX,Scale=1}
\fi
\usepackage{lmodern}
\ifPDFTeX\else
  % xetex/luatex font selection
\fi
% Use upquote if available, for straight quotes in verbatim environments
\IfFileExists{upquote.sty}{\usepackage{upquote}}{}
\IfFileExists{microtype.sty}{% use microtype if available
  \usepackage[]{microtype}
  \UseMicrotypeSet[protrusion]{basicmath} % disable protrusion for tt fonts
}{}
\makeatletter
\@ifundefined{KOMAClassName}{% if non-KOMA class
  \IfFileExists{parskip.sty}{%
    \usepackage{parskip}
  }{% else
    \setlength{\parindent}{0pt}
    \setlength{\parskip}{6pt plus 2pt minus 1pt}}
}{% if KOMA class
  \KOMAoptions{parskip=half}}
\makeatother
\usepackage{xcolor}
\usepackage[margin=1in]{geometry}
\usepackage{color}
\usepackage{fancyvrb}
\newcommand{\VerbBar}{|}
\newcommand{\VERB}{\Verb[commandchars=\\\{\}]}
\DefineVerbatimEnvironment{Highlighting}{Verbatim}{commandchars=\\\{\}}
% Add ',fontsize=\small' for more characters per line
\usepackage{framed}
\definecolor{shadecolor}{RGB}{248,248,248}
\newenvironment{Shaded}{\begin{snugshade}}{\end{snugshade}}
\newcommand{\AlertTok}[1]{\textcolor[rgb]{0.94,0.16,0.16}{#1}}
\newcommand{\AnnotationTok}[1]{\textcolor[rgb]{0.56,0.35,0.01}{\textbf{\textit{#1}}}}
\newcommand{\AttributeTok}[1]{\textcolor[rgb]{0.13,0.29,0.53}{#1}}
\newcommand{\BaseNTok}[1]{\textcolor[rgb]{0.00,0.00,0.81}{#1}}
\newcommand{\BuiltInTok}[1]{#1}
\newcommand{\CharTok}[1]{\textcolor[rgb]{0.31,0.60,0.02}{#1}}
\newcommand{\CommentTok}[1]{\textcolor[rgb]{0.56,0.35,0.01}{\textit{#1}}}
\newcommand{\CommentVarTok}[1]{\textcolor[rgb]{0.56,0.35,0.01}{\textbf{\textit{#1}}}}
\newcommand{\ConstantTok}[1]{\textcolor[rgb]{0.56,0.35,0.01}{#1}}
\newcommand{\ControlFlowTok}[1]{\textcolor[rgb]{0.13,0.29,0.53}{\textbf{#1}}}
\newcommand{\DataTypeTok}[1]{\textcolor[rgb]{0.13,0.29,0.53}{#1}}
\newcommand{\DecValTok}[1]{\textcolor[rgb]{0.00,0.00,0.81}{#1}}
\newcommand{\DocumentationTok}[1]{\textcolor[rgb]{0.56,0.35,0.01}{\textbf{\textit{#1}}}}
\newcommand{\ErrorTok}[1]{\textcolor[rgb]{0.64,0.00,0.00}{\textbf{#1}}}
\newcommand{\ExtensionTok}[1]{#1}
\newcommand{\FloatTok}[1]{\textcolor[rgb]{0.00,0.00,0.81}{#1}}
\newcommand{\FunctionTok}[1]{\textcolor[rgb]{0.13,0.29,0.53}{\textbf{#1}}}
\newcommand{\ImportTok}[1]{#1}
\newcommand{\InformationTok}[1]{\textcolor[rgb]{0.56,0.35,0.01}{\textbf{\textit{#1}}}}
\newcommand{\KeywordTok}[1]{\textcolor[rgb]{0.13,0.29,0.53}{\textbf{#1}}}
\newcommand{\NormalTok}[1]{#1}
\newcommand{\OperatorTok}[1]{\textcolor[rgb]{0.81,0.36,0.00}{\textbf{#1}}}
\newcommand{\OtherTok}[1]{\textcolor[rgb]{0.56,0.35,0.01}{#1}}
\newcommand{\PreprocessorTok}[1]{\textcolor[rgb]{0.56,0.35,0.01}{\textit{#1}}}
\newcommand{\RegionMarkerTok}[1]{#1}
\newcommand{\SpecialCharTok}[1]{\textcolor[rgb]{0.81,0.36,0.00}{\textbf{#1}}}
\newcommand{\SpecialStringTok}[1]{\textcolor[rgb]{0.31,0.60,0.02}{#1}}
\newcommand{\StringTok}[1]{\textcolor[rgb]{0.31,0.60,0.02}{#1}}
\newcommand{\VariableTok}[1]{\textcolor[rgb]{0.00,0.00,0.00}{#1}}
\newcommand{\VerbatimStringTok}[1]{\textcolor[rgb]{0.31,0.60,0.02}{#1}}
\newcommand{\WarningTok}[1]{\textcolor[rgb]{0.56,0.35,0.01}{\textbf{\textit{#1}}}}
\usepackage{graphicx}
\makeatletter
\def\maxwidth{\ifdim\Gin@nat@width>\linewidth\linewidth\else\Gin@nat@width\fi}
\def\maxheight{\ifdim\Gin@nat@height>\textheight\textheight\else\Gin@nat@height\fi}
\makeatother
% Scale images if necessary, so that they will not overflow the page
% margins by default, and it is still possible to overwrite the defaults
% using explicit options in \includegraphics[width, height, ...]{}
\setkeys{Gin}{width=\maxwidth,height=\maxheight,keepaspectratio}
% Set default figure placement to htbp
\makeatletter
\def\fps@figure{htbp}
\makeatother
\setlength{\emergencystretch}{3em} % prevent overfull lines
\providecommand{\tightlist}{%
  \setlength{\itemsep}{0pt}\setlength{\parskip}{0pt}}
\setcounter{secnumdepth}{-\maxdimen} % remove section numbering
\ifLuaTeX
  \usepackage{selnolig}  % disable illegal ligatures
\fi
\usepackage{bookmark}
\IfFileExists{xurl.sty}{\usepackage{xurl}}{} % add URL line breaks if available
\urlstyle{same}
\hypersetup{
  pdftitle={Datenprojekt1},
  pdfauthor={R.C. Wallner},
  hidelinks,
  pdfcreator={LaTeX via pandoc}}

\title{Datenprojekt1}
\author{R.C. Wallner}
\date{2024-11-22}

\begin{document}
\maketitle

\begin{Shaded}
\begin{Highlighting}[]
\DocumentationTok{\#\#\#1}
\NormalTok{df1 }\OtherTok{\textless{}{-}} \FunctionTok{data.frame}\NormalTok{(}\AttributeTok{Betrieb =}\NormalTok{ data}\SpecialCharTok{$}\NormalTok{ID\_betnr,}\AttributeTok{HO =}\NormalTok{ data}\SpecialCharTok{$}\NormalTok{bhomeoff)}
\NormalTok{df2 }\OtherTok{\textless{}{-}} \FunctionTok{data.frame}\NormalTok{(}\AttributeTok{Betrieb =} \DecValTok{1}\NormalTok{, }\AttributeTok{HO =} \DecValTok{0}\NormalTok{) }\CommentTok{\#initialisieren mit Erstem Betrieb}
\ControlFlowTok{for}\NormalTok{(i }\ControlFlowTok{in} \DecValTok{1}\SpecialCharTok{:}\FunctionTok{nrow}\NormalTok{(df1))\{}
  \ControlFlowTok{if}\NormalTok{( }\SpecialCharTok{!}\NormalTok{ (  df1[i,}\DecValTok{1}\NormalTok{] }\SpecialCharTok{\%in\%}\NormalTok{ df2[,}\DecValTok{1}\NormalTok{] ) )\{}
\NormalTok{    df2 }\OtherTok{\textless{}{-}} \FunctionTok{rbind}\NormalTok{(df2, df1[i,])}
\NormalTok{  \}}
\NormalTok{\}}
\CommentTok{\#Wenn die Betriebnummer noch nicht in der bisherigen Spalte 1 des neuen}
\CommentTok{\#Datenrahmens aufgenommen wurde, dann wird die Zeile hinzugefügt }

\FunctionTok{nrow}\NormalTok{(df2) }\SpecialCharTok{==} \FunctionTok{length}\NormalTok{(}\FunctionTok{unique}\NormalTok{(df1[,}\DecValTok{1}\NormalTok{])) }\CommentTok{\#TRUE heißt alle Betirb{-}Duplikate eliminert}
\end{Highlighting}
\end{Shaded}

\begin{verbatim}
## [1] TRUE
\end{verbatim}

\begin{Shaded}
\begin{Highlighting}[]
\NormalTok{df2 }\OtherTok{\textless{}{-}}\NormalTok{ df2[}\SpecialCharTok{{-}}\FunctionTok{nrow}\NormalTok{(df2), ] }\CommentTok{\#Die NA{-}Betriebe kommen auch 1 mal ganz am Ende vor}
\NormalTok{df2}\SpecialCharTok{$}\NormalTok{HO }\OtherTok{\textless{}{-}} \FunctionTok{factor}\NormalTok{(df2}\SpecialCharTok{$}\NormalTok{HO, }\AttributeTok{levels=}\FunctionTok{c}\NormalTok{(}\DecValTok{0}\NormalTok{,}\DecValTok{1}\NormalTok{), }\AttributeTok{labels =} \FunctionTok{c}\NormalTok{(}\StringTok{"Nein"}\NormalTok{,}\StringTok{"Ja"}\NormalTok{))}
\NormalTok{Frequency\_Table }\OtherTok{\textless{}{-}} \FunctionTok{prop.table}\NormalTok{(}\FunctionTok{table}\NormalTok{(df2}\SpecialCharTok{$}\NormalTok{HO))}
\NormalTok{Share\_HO }\OtherTok{\textless{}{-}} \FunctionTok{as.numeric}\NormalTok{(Frequency\_Table[}\DecValTok{2}\NormalTok{])}
\NormalTok{Share\_HO}
\end{Highlighting}
\end{Shaded}

\begin{verbatim}
## [1] 0.3653137
\end{verbatim}

\begin{Shaded}
\begin{Highlighting}[]
\FunctionTok{barplot}\NormalTok{(Frequency\_Table, }\AttributeTok{main=}\StringTok{"Erlauben Betriebe Home Office?"}\NormalTok{,}
        \AttributeTok{col=}\FunctionTok{c}\NormalTok{(}\StringTok{"darkred"}\NormalTok{,}\StringTok{"darkgreen"}\NormalTok{), }\AttributeTok{horiz =}\NormalTok{ T)}
\end{Highlighting}
\end{Shaded}

\includegraphics{Markdown1_files/figure-latex/unnamed-chunk-2-1.pdf}

\begin{Shaded}
\begin{Highlighting}[]
\DocumentationTok{\#\#\#2}
\NormalTok{df3 }\OtherTok{\textless{}{-}} \FunctionTok{cbind}\NormalTok{(df1, }\AttributeTok{Nutzen =}\NormalTok{ data}\SpecialCharTok{$}\NormalTok{mheim)[df1}\SpecialCharTok{$}\NormalTok{HO}\SpecialCharTok{==}\DecValTok{1}\NormalTok{,]}
\NormalTok{df4 }\OtherTok{\textless{}{-}} \FunctionTok{data.frame}\NormalTok{(}\AttributeTok{Betrieb =} \DecValTok{6}\NormalTok{, }\AttributeTok{HO =} \DecValTok{1}\NormalTok{, }\AttributeTok{Nutzen =} \DecValTok{1}\NormalTok{)}
\ControlFlowTok{for}\NormalTok{(i }\ControlFlowTok{in} \DecValTok{2}\SpecialCharTok{:}\FunctionTok{nrow}\NormalTok{(df3))\{}
\NormalTok{  enthält }\OtherTok{\textless{}{-}}\NormalTok{ F}
  \ControlFlowTok{for}\NormalTok{(j }\ControlFlowTok{in} \DecValTok{1}\SpecialCharTok{:}\DecValTok{3}\NormalTok{)\{}
    \ControlFlowTok{if}\NormalTok{( }\FunctionTok{is.na}\NormalTok{(df3[i,j]) )\{enthält }\OtherTok{\textless{}{-}}\NormalTok{ T\}}
\NormalTok{  \}}
  \ControlFlowTok{if}\NormalTok{( }\SpecialCharTok{!}\NormalTok{enthält)\{}
\NormalTok{    df4 }\OtherTok{\textless{}{-}} \FunctionTok{rbind}\NormalTok{(df4, df3[i,])}
\NormalTok{  \}}
\NormalTok{\}}
\NormalTok{df4}\SpecialCharTok{$}\NormalTok{Nutzen }\OtherTok{\textless{}{-}} \FunctionTok{factor}\NormalTok{(df4}\SpecialCharTok{$}\NormalTok{Nutzen, }\AttributeTok{levels=}\FunctionTok{c}\NormalTok{(}\DecValTok{0}\NormalTok{,}\DecValTok{1}\NormalTok{), }\AttributeTok{labels=} \FunctionTok{c}\NormalTok{(}
  \StringTok{"Nutzt Möglichkeit nicht"}\NormalTok{, }\StringTok{"Nutzt Möglichkeit"}
\NormalTok{))}
\NormalTok{Frequency\_Use }\OtherTok{\textless{}{-}} \FunctionTok{prop.table}\NormalTok{(}\FunctionTok{table}\NormalTok{(df4}\SpecialCharTok{$}\NormalTok{Nutzen))}
\NormalTok{Frequency\_Use }\CommentTok{\#Knapp 1/4 genau 23.63\% nutzen die HO Möglichkeit}
\end{Highlighting}
\end{Shaded}

\begin{verbatim}
## 
## Nutzt Möglichkeit nicht       Nutzt Möglichkeit 
##               0.7636816               0.2363184
\end{verbatim}

\begin{Shaded}
\begin{Highlighting}[]
\FunctionTok{barplot}\NormalTok{(Frequency\_Use, }\AttributeTok{col=}\FunctionTok{c}\NormalTok{(}\StringTok{"darkblue"}\NormalTok{, }\StringTok{"lightblue"}\NormalTok{),}
        \AttributeTok{main=} \StringTok{"Nutzt der Mitarbeiter die Möglichkeit zum Home Office?"}\NormalTok{)}
\end{Highlighting}
\end{Shaded}

\includegraphics{Markdown1_files/figure-latex/unnamed-chunk-3-1.pdf}

\begin{Shaded}
\begin{Highlighting}[]
\DocumentationTok{\#\#\#3}
\NormalTok{Relevante\_Variablen }\OtherTok{\textless{}{-}} \FunctionTok{character}\NormalTok{()}
\ControlFlowTok{for}\NormalTok{(i }\ControlFlowTok{in} \DecValTok{1}\SpecialCharTok{:}\FunctionTok{length}\NormalTok{(}\FunctionTok{colnames}\NormalTok{(data)))\{}
  \ControlFlowTok{if}\NormalTok{( }\FunctionTok{grepl}\NormalTok{(}\StringTok{"mheimnein\_"}\NormalTok{, }\FunctionTok{colnames}\NormalTok{(data)[i] )   )\{}
\NormalTok{    Relevante\_Variablen }\OtherTok{\textless{}{-}} \FunctionTok{c}\NormalTok{( Relevante\_Variablen, }\FunctionTok{colnames}\NormalTok{(data)[i])}
\NormalTok{  \}}
\NormalTok{\} }\CommentTok{\#alle Variablen, die mheimnein\_ enthalten, also hier zweckdienlich sind}
\NormalTok{Index }\OtherTok{\textless{}{-}} \FunctionTok{which}\NormalTok{(}\FunctionTok{colnames}\NormalTok{(data)  }\SpecialCharTok{\%in\%}\NormalTok{ Relevante\_Variablen )}
\NormalTok{df5 }\OtherTok{\textless{}{-}} \FunctionTok{na.omit}\NormalTok{( }\FunctionTok{cbind}\NormalTok{(df1, data}\SpecialCharTok{$}\NormalTok{mheim, data[,Index]) )}
\NormalTok{nj }\OtherTok{\textless{}{-}} \FunctionTok{numeric}\NormalTok{()}
\ControlFlowTok{for}\NormalTok{(j }\ControlFlowTok{in} \DecValTok{4}\SpecialCharTok{:}\FunctionTok{ncol}\NormalTok{(df5))\{}
\NormalTok{  nj }\OtherTok{\textless{}{-}} \FunctionTok{c}\NormalTok{(nj, }\FunctionTok{sum}\NormalTok{(df5[,j]) )}
\NormalTok{\}}
\NormalTok{Relevante\_Variablen2 }\OtherTok{\textless{}{-}} \FunctionTok{c}\NormalTok{(}\StringTok{"Erl."}\NormalTok{, }\StringTok{"Tech."}\NormalTok{, }\StringTok{"unmö."}\NormalTok{, }\StringTok{"vorg."}\NormalTok{, }\StringTok{"trenn."}\NormalTok{, }\StringTok{"team"}\NormalTok{, }\StringTok{"karr."}\NormalTok{)}
\NormalTok{Gründe }\OtherTok{\textless{}{-}} \FunctionTok{data.frame}\NormalTok{(Relevante\_Variablen2, nj)[}\FunctionTok{order}\NormalTok{(nj), ]}
\FunctionTok{barplot}\NormalTok{(Gründe}\SpecialCharTok{$}\NormalTok{nj, }\AttributeTok{names.arg=}\NormalTok{ Gründe}\SpecialCharTok{$}\NormalTok{Relevante\_Variablen2,}
        \AttributeTok{col=}\FunctionTok{c}\NormalTok{(}\FunctionTok{rep}\NormalTok{(}\StringTok{"darkblue"}\NormalTok{,}\DecValTok{2}\NormalTok{),}\FunctionTok{rep}\NormalTok{(}\StringTok{"blue"}\NormalTok{,}\DecValTok{3}\NormalTok{),}\FunctionTok{rep}\NormalTok{(}\StringTok{"lightblue"}\NormalTok{,}\DecValTok{2}\NormalTok{)) ,}
        \AttributeTok{main=}\StringTok{"Grund für kein Home Office"}\NormalTok{)}
\end{Highlighting}
\end{Shaded}

\includegraphics{Markdown1_files/figure-latex/unnamed-chunk-4-1.pdf}

\begin{Shaded}
\begin{Highlighting}[]
\CommentTok{\#Die meisten geben einfach nur an, dass es Ihnen nicht möglich ist,}
\CommentTok{\#am zweit häufigsten ist der Grund, dass Vorgesetzten die Anwesenheit "sehr wichtig" ist}
\end{Highlighting}
\end{Shaded}

\begin{Shaded}
\begin{Highlighting}[]
\DocumentationTok{\#\#\#4}
\NormalTok{df6 }\OtherTok{\textless{}{-}} \FunctionTok{factor}\NormalTok{(}\FunctionTok{na.omit}\NormalTok{(}\FunctionTok{subset}\NormalTok{( }\FunctionTok{data.frame}\NormalTok{( }
                         \AttributeTok{wunsch3 =}\NormalTok{ data}\SpecialCharTok{$}\NormalTok{mheimwunsch, }
                         \AttributeTok{HO =}\NormalTok{ data}\SpecialCharTok{$}\NormalTok{bhomeoff),}
\NormalTok{              HO }\SpecialCharTok{==} \DecValTok{0}\NormalTok{))[,}\DecValTok{1}\NormalTok{],}\AttributeTok{levels=}\DecValTok{1}\SpecialCharTok{:}\DecValTok{3}\NormalTok{, }\AttributeTok{labels =} \FunctionTok{c}\NormalTok{(}\FunctionTok{rep}\NormalTok{(}\StringTok{"Wunsch"}\NormalTok{,}\DecValTok{2}\NormalTok{),}\StringTok{"Kein Wunsch"}\NormalTok{) )}
\CommentTok{\#df6 ist ein Vektor der alle mheimwunsch für HO = 0 herausfindet, und diese dann in}
\CommentTok{\#Wunscht besteht: Werte 1,2 und Wunsch besteht nicht: Wert 3 aufteilt}
\FunctionTok{barplot}\NormalTok{(}\FunctionTok{table}\NormalTok{(df6), }\AttributeTok{main =} \StringTok{"wünschen sich Arbeitnehmer }\SpecialCharTok{\textbackslash{}n}\StringTok{ in Betrieben, die kein HO anbieten}
\StringTok{        von, zu Hause aus zu arbeiten?"}\NormalTok{,}
        \AttributeTok{col=}\FunctionTok{c}\NormalTok{(}\StringTok{"lightblue"}\NormalTok{,}\StringTok{"darkblue"}\NormalTok{))}
\end{Highlighting}
\end{Shaded}

\includegraphics{Markdown1_files/figure-latex/unnamed-chunk-5-1.pdf}

\begin{Shaded}
\begin{Highlighting}[]
\DocumentationTok{\#\#\#5}
\NormalTok{Wichtige\_Variablen }\OtherTok{\textless{}{-}} \FunctionTok{character}\NormalTok{()}
\ControlFlowTok{for}\NormalTok{(i }\ControlFlowTok{in} \DecValTok{1}\SpecialCharTok{:}\FunctionTok{ncol}\NormalTok{(data))\{}
  \ControlFlowTok{if}\NormalTok{( }\FunctionTok{grepl}\NormalTok{(}\StringTok{"mheimwunsch\_"}\NormalTok{, }\FunctionTok{colnames}\NormalTok{(data)[i]))\{}
\NormalTok{    Wichtige\_Variablen }\OtherTok{\textless{}{-}} \FunctionTok{c}\NormalTok{(Wichtige\_Variablen, }\FunctionTok{colnames}\NormalTok{(data)[i])}
\NormalTok{  \}}
\NormalTok{\}}
\NormalTok{Index2 }\OtherTok{\textless{}{-}} \FunctionTok{which}\NormalTok{(}\FunctionTok{colnames}\NormalTok{(data) }\SpecialCharTok{\%in\%}\NormalTok{ Wichtige\_Variablen)}
\NormalTok{df7 }\OtherTok{\textless{}{-}} \FunctionTok{na.omit}\NormalTok{(data[, Index2])}
\NormalTok{Gründe2 }\OtherTok{\textless{}{-}} \FunctionTok{c}\NormalTok{(}\StringTok{"fahren"}\NormalTok{,}\StringTok{"freiz."}\NormalTok{,}\StringTok{"fam"}\NormalTok{,}\StringTok{"qual."}\NormalTok{,}\StringTok{"h\_erhöhen"}\NormalTok{)}
\NormalTok{Häufigkeiten }\OtherTok{\textless{}{-}} \FunctionTok{apply}\NormalTok{(df7, }\DecValTok{2}\NormalTok{, sum)}
\FunctionTok{barplot}\NormalTok{(Häufigkeiten, }\AttributeTok{names.arg=}\NormalTok{Gründe2,}
        \AttributeTok{col=}\StringTok{"lightblue"}\NormalTok{)}
\end{Highlighting}
\end{Shaded}

\includegraphics{Markdown1_files/figure-latex/unnamed-chunk-6-1.pdf}

\end{document}
